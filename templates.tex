\section{Related Work}
\label{sec:relatedwork}

Winery~\cite{Winery} is a graphical \commentontext{modeling}{modeling with one ``l'', because of AE} tool.
The whole idea of TOSCA is explained by \citet{Binz2009}.

\section{LaTeX Hints}
\label{sec:latexhints}

% Required for proper example rendering in the compiled PDF
\newcount\LTGbeginlineexample
\newcount\LTGendlineexample
\newenvironment{ltgexample}%
{\LTGbeginlineexample=\numexpr\inputlineno+1\relax}%
{\LTGendlineexample=\numexpr\inputlineno-1\relax%
%
\tcbinputlisting{%
  listing only,
  listing file=\currfilepath,
  colback=green!5!white,
  colframe=green!25,
  coltitle=black!90,
  coltext=black!90,
  left=8mm,
  title=Corresponding \LaTeX{} code of \texttt{\currfilepath},
  listing options={%
    frame=none,
    language={[LaTeX]TeX},
    escapeinside={},
    firstline=\the\LTGbeginlineexample,
    lastline=\the\LTGendlineexample,
    firstnumber=\the\LTGbeginlineexample,
    basewidth=.5em,
    aboveskip=0mm,
    belowskip=0mm,
    numbers=left,
    xleftmargin=0mm,
    numberstyle=\tiny,
    numbersep=8pt%
  }
}
}%

This section contains hints on writing LaTeX.
It focuses on minimal examples, which can be directly adapted to the content

\subsection{Handling of paragraphs}

\begin{ltgexample}
One sentence per line.
This rule is important for the usage of version control systems.
A new line is generated with a blank line.
As you would do in Word:
New paragraphs are generated by pressing enter.
In LaTeX, this does not lead to a new paragraph as LaTeX joins subsequent lines.
In case you want a new paragraph, just press enter twice (!).
This leads to an empty line.
In word, there is the functionality to press shift and enter.
This leads to a hard line break.
The text starts at the beginning of a new line.
In LaTeX, you can do that by using two backslashes (\textbackslash\textbackslash).\\
This is rarely used.

Please do \textit{not} use two backslashes for new paragraphs.
For instance, this sentence belongs to the same paragraph, whereas the last one started a new one.
A long motivation for that is provided at \url{http://loopspace.mathforge.org/HowDidIDoThat/TeX/VCS/#section.3}.
\end{ltgexample}

\subsection{Notes separated from the text}

The package mindflow enables writing down notes and annotations in a way so that they are separated from the main text.

\begin{ltgexample}
\begin{mindflow}
This is a small note.
\end{mindflow}
\end{ltgexample}

\subsection{Hyphenation}

\LaTeX{} automatically hyphenates words.
When using microtype, there should be less hypnetations than in other settings.
It might be necessary to tweak the hyphenations nevertheless.
Here are some hints:

\begin{ltgexample}
In case you write \enquote{application-specific}, then the word will only be hyphenated at the dash.
You can also write \verb1applica\allowbreak{}tion-specific1 (result: applica\allowbreak{}tion-specific), but this is much more effort.

You can now write words containing hyphens which are hyphenated at other places in the word.
For instance, \verb1application"=specific1 gets application"=specific.
This is enabled by an additional configuration of the babel package.
\end{ltgexample}

\subsection{Typesetting Units}

\begin{ltgexample}
Numbers can written plain text (such as 100), by using the siunitx package like that:
\SI{100}{\km\per\hour},
or by using plain \LaTeX{} (and math mode):
$100 \frac{\mathit{km}}{h}$.
\end{ltgexample}

\begin{ltgexample}
\SI{5}{\percent} of \SI{10}{kg}
\end{ltgexample}

\begin{ltgexample}
Numbers are automatically grouped: \num{123456}.
\end{ltgexample}

\subsection{Surrounding Text by Quotes}

\begin{ltgexample}
Please use the \enquote{enquote command} to quote something.
Quoting with "`quote"' or ``quote'' also works.

\end{ltgexample}

\subsection{Cleveref examples}
\label{sec:ex:cref}

Cleveref demonstration: Cref at beginning of sentence, cref in all other cases.

\begin{figure}
    \centering
    \includegraphics[width=.75\linewidth]{example-image-a}
    \caption{Example figure for cref demo}
    \label{fig:ex:cref}
\end{figure}

\begin{table}
    \centering
    \begin{tabular}{ll}
      \toprule
      Heading1 & Heading2 \\
      \midrule
      One      & Two      \\
      Thee     & Four     \\
      \bottomrule
    \end{tabular}
    \caption{Example table for cref demo}
    \label{tab:ex:cref}
\end{table}

\begin{ltgexample}
\Cref{fig:ex:cref} shows a simple fact, although \cref{fig:ex:cref} could also show something else.

\Cref{tab:ex:cref} shows a simple fact, although \cref{tab:ex:cref} could also show something else.

\Cref{sec:ex:cref} shows a simple fact, although \cref{sec:ex:cref} could also show something else.
\end{ltgexample}

\subsection{Figures}

\begin{ltgexample}
\Cref{fig:label} shows something interesting.

\begin{figure}
  \centering
  \includegraphics[width=.8\linewidth]{example-image-golden}
  \caption[Simple Figure]{Simple Figure. Based on \citet{mwe}.}
  \label{fig:label}
\end{figure}
\end{ltgexample}

One can also have pictures floating inside text:
\clearpage

\begin{ltgexample}
\begin{floatingfigure}{.33\linewidth}
\includegraphics[width=.29\linewidth]{example-image-a}
\caption{A floating figure}
\end{floatingfigure}
\blindtext[2]
\end{ltgexample}


\subsection{Sub Figures}

An example of two sub figures is shown in \cref{fig:two_sub_figures}.

\begin{ltgexample}
\begin{figure}[!b]
    \centering
    \subfloat[Case I]{\includegraphics[width=.4\linewidth]{example-image-a}%
    \label{fig:first_case}}
  \hfil
    \subfloat[Case II]{\includegraphics[width=.4\linewidth]{example-image-b}%
    \label{fig:second_case}}
  \caption{Example figure with two sub figures.}
  \label{fig:two_sub_figures}
\end{figure}
\end{ltgexample}

\subsection{Tables}

\begin{ltgexample}
\begin{table}
  \caption{Simple Table}
  \label{tab:simple}
  \centering
  \begin{tabular}{ll}
    \toprule
    Heading1 & Heading2 \\
    \midrule
    One      & Two      \\
    Thee     & Four     \\
    \bottomrule
  \end{tabular}
\end{table}
\end{ltgexample}

\begin{ltgexample}
% Source: https://tex.stackexchange.com/a/468994/9075
\begin{table}
\caption{Table with diagonal line}
\label{tab:diag}
\begin{center}
\begin{tabular}{|l|c|c|}
\hline
\diagbox[width=10em]{Diag\\Column Head I}{Diag Column\\Head II} & Second & Third \\
\hline
& foo & bar \\
\hline
\end{tabular}
\end{center}
\end{table}
\end{ltgexample}


\subsection{Source Code}

\begin{ltgexample}
\Cref{lst:XML} shows source code written in XML.
\Cref{line:comment} contains a comment.

\begin{lstlisting}[
  language=XML,
  caption={Example XML Listing},
  label={lst:XML}]
<listing name="example">
  <!-- comment --> (* \label{line:comment} *)
  <content>not interesting</content>
</listing>
\end{lstlisting}
\end{ltgexample}

One can also add \verb+float+ as parameter to have the listing floating.
\Cref{lst:flXML} shows the floating listing.

\begin{ltgexample}
\begin{lstlisting}[
  % one can adjust spacing here if required
  % aboveskip=2.5\baselineskip,
  % belowskip=-.8\baselineskip,
  float,
  language=XML,
  caption={Example XML listing -- placed as floating figure},
  label={lst:flXML}]
<listing name="example">
  Floating
</listing>
\end{lstlisting}
\end{ltgexample}

One can also typeset JSON as shown in \cref{lst:json}.

\begin{ltgexample}
\begin{lstlisting}[
  float,
  language=json,
  caption={Example JSON listing -- placed as floating figure},
  label={lst:json}]
{
  key: "value"
}
\end{lstlisting}
\end{ltgexample}

Java is also possible as shown in \cref{lst:java}.

\begin{ltgexample}
\begin{lstlisting}[
  caption={Example Java listing},
  label=lst:java,
  language=Java,
  float]
public class Hello {
    public static void main (String[] args) {
        System.out.println("Hello World!");
    }
}
\end{lstlisting}
\end{ltgexample}

\subsection{Itemization}

One can list items as follows:

\begin{ltgexample}
\begin{itemize}
\item Item One
\item Item Two
\end{itemize}
\end{ltgexample}


One can enumerate items as follows:

\begin{ltgexample}
\begin{enumerate}
  \item Item One
  \item Item Two
\end{enumerate}
\end{ltgexample}


With paralist, one can even have all items typset after each other and have them clean in the tex document:

\begin{ltgexample}
\begin{inparaenum}
  \item All these items...
  \item ...appear in one line
  \item This is enabled by the paralist package.
\end{inparaenum}
\end{ltgexample}

\subsection{Other Features}

\begin{ltgexample}
The words \enquote{workflow} and \enquote{dwarflike} can be copied from the PDF and pasted to a text file.
\end{ltgexample}

\begin{ltgexample}
The symbol for powerset is now correct: $\powerset$ and not a Weierstrass p ($\wp$).

$\powerset({1,2,3})$
\end{ltgexample}

\begin{ltgexample}
Brackets work as designed:
<test>
One can also input backquotes in verbatim text: \verb|`test`|.
\end{ltgexample}


\section{Conclusion and Outlook}
\label{sec:outlook}
\lipsum[1-2]

\subsubsection*{Acknowledgments}

Identification of funding sources and other support, and thanks to individuals and groups that assisted in the research and the preparation of the work should be included in an acknowledgment section, which is placed just before the reference section in your document \cite{acmart}.

%%% ===============================================================================
%%% Bibliography
%%% ===============================================================================

In the bibliography, use \texttt{\textbackslash textsuperscript} for \enquote{st}, \enquote{nd}, \ldots:
E.g., \enquote{The 2\textsuperscript{nd} conference on examples}.
When you use \href{https://www.jabref.org}{JabRef}, you can use the clean up command to achieve that.
See \url{https://help.jabref.org/en/CleanupEntries} for an overview of the cleanup functionality.

\renewcommand{\bibsection}{\section*{References}} % requried for natbib to have "References" printed and as section*, not chapter*
% Use natbib compatbile splncs04nat style.
% It does provide all features of splncs04.bst, but is developed in a clean way.
% Source: https://github.com/tpavlic/splncs04nat
\bibliographystyle{splncs04nat}
\begingroup
  \microtypecontext{expansion=sloppy}
  \small % ensure correct font size for the bibliography
  \bibliography{paper}
\endgroup

% Enfore empty line after bibliography
\ \\
%
All links were last followed on October 5, 2020.

%%% ===============================================================================
%\appendix
%\addcontentsline{toc}{chapter}{APPENDICES}

%\listoffigures
%\listoftables
%%% ===============================================================================

%%% ===============================================================================
%\section{My first appendix}\label{sec:appendix1}
%%% ===============================================================================