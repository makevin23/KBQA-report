% This template has been tested with LLNCS DOCUMENT CLASS -- version 2.21 (12-Jan-2022)

% !TeX spellcheck = en-US
% LTeX: language=en-US
% !TeX encoding = utf8
% !TeX program = pdflatex
% !BIB program = bibtex
% -*- coding:utf-8 mod:LaTeX -*-

% "a4paper" enables:
%
%  - easy print out on DIN A4 paper size
%
% One can configure a4 vs. letter in the LaTeX installation. So it is configuration dependend, what the paper size will be.
% This option  present, because the current word template offered by Springer is DIN A4.
% We accept that DIN A4 cause WTFs at persons not used to A4 in USA.
%
% "runningheads" enables:
%
%  - page number on page 2 onwards
%  - title/authors on even/odd pages
%
% This is good for other readers to enable proper archiving among other papers and pointing to
% content. Even if the title page states the title, when printed and stored in a folder, when
% blindly opening the folder, one could hit not the title page, but an arbitrary page. Therefore,
% it is good to have title printed on the pages, too.
%
% To disable outputting page headers and footers, remove "runningheads"
\documentclass[runningheads,a4paper,english]{llncs}[2022/01/12]

% backticks (`) are rendered as such in verbatim environments.
% See following links for details:
%   - https://tex.stackexchange.com/a/341057/9075
%   - https://tex.stackexchange.com/a/47451/9075
%   - https://tex.stackexchange.com/a/166791/9075
\usepackage{upquote}

% Set English as language and allow to write hyphenated"=words
%
% Even though `american`, `english` and `USenglish` are synonyms for babel package (according to https://tex.stackexchange.com/questions/12775/babel-english-american-usenglish), the llncs document class is prepared to avoid the overriding of certain names (such as "Abstract." -> "Abstract" or "Fig." -> "Figure") when using `english`, but not when using the other 2.
% english has to go last to set it as default language
\usepackage[ngerman,main=english]{babel}
%
% Hint by http://tex.stackexchange.com/a/321066/9075 -> enable "= as dashes
\addto\extrasenglish{\languageshorthands{ngerman}\useshorthands{"}}
%
% Fix by https://tex.stackexchange.com/a/441701/9075
\usepackage{regexpatch}
\makeatletter
\edef\switcht@albion{%
  \relax\unexpanded\expandafter{\switcht@albion}%
}
\xpatchcmd*{\switcht@albion}{ \def}{\def}{}{}
\xpatchcmd{\switcht@albion}{\relax}{}{}{}
\edef\switcht@deutsch{%
  \relax\unexpanded\expandafter{\switcht@deutsch}%
}
\xpatchcmd*{\switcht@deutsch}{ \def}{\def}{}{}
\xpatchcmd{\switcht@deutsch}{\relax}{}{}{}
\edef\switcht@francais{%
  \relax\unexpanded\expandafter{\switcht@francais}%
}
\xpatchcmd*{\switcht@francais}{ \def}{\def}{}{}
\xpatchcmd{\switcht@francais}{\relax}{}{}{}
\makeatother

% Links behave as they should. Enables "\url{...}" for URL typesettings.
% Allow URL breaks also at a hyphen, even though it might be confusing: Is the "-" part of the address or just a hyphen?
% See https://tex.stackexchange.com/a/3034/9075.
\usepackage[hyphens]{url}

% When activated, use text font as url font, not the monospaced one.
% For all options see https://tex.stackexchange.com/a/261435/9075.
% \urlstyle{same}

% Improve wrapping of URLs - hint by http://tex.stackexchange.com/a/10419/9075
\makeatletter
\g@addto@macro{\UrlBreaks}{\UrlOrds}
\makeatother

% nicer // - solution by http://tex.stackexchange.com/a/98470/9075
% DO NOT ACTIVATE -> prevents line breaks
%\makeatletter
%\def\Url@twoslashes{\mathchar`\/\@ifnextchar/{\kern-.2em}{}}
%\g@addto@macro\UrlSpecials{\do\/{\Url@twoslashes}}
%\makeatother

% This is the modern package for "Computer Modern".
% In case this gets activated, one has to switch from cmap package to glyphtounicode (in the case of pdflatex)
\usepackage[%
    rm={oldstyle=false,proportional=true},%
    sf={oldstyle=false,proportional=true},%
    % By using 'variable=true' the monospaced font can be used as variable font (with differents widths per letter)
    % However, this makes listings look ugly.
    tt={oldstyle=false,proportional=true,variable=false},%
    qt=false%
]{cfr-lm}

% Has to be loaded AFTER any font packages. See https://tex.stackexchange.com/a/2869/9075.
\usepackage[T1]{fontenc}

% Character protrusion and font expansion. See http://www.ctan.org/tex-archive/macros/latex/contrib/microtype/

\usepackage[
  babel=true, % Enable language-specific kerning. Take language-settings from the languge of the current document (see Section 6 of microtype.pdf)
  expansion=alltext,
  protrusion=alltext-nott, % Ensure that at listings, there is no change at the margin of the listing
  final % Always enable microtype, even if in draft mode. This helps finding bad boxes quickly.
        % In the standard configuration, this template is always in the final mode, so this option only makes a difference if "pros" use the draft mode
]{microtype}

% \texttt{test -- test} keeps the "--" as "--" (and does not convert it to an en dash)
\DisableLigatures{encoding = T1, family = tt* }

%\DeclareMicrotypeSet*[tracking]{my}{ font = */*/*/sc/* }%
%\SetTracking{ encoding = *, shape = sc }{ 45 }
% Source: http://homepage.ruhr-uni-bochum.de/Georg.Verweyen/pakete.html
% Deactiviated, because does not look good

\usepackage{graphicx}

% Diagonal lines in a table - http://tex.stackexchange.com/questions/17745/diagonal-lines-in-table-cell
% Slashbox is not available in texlive (due to licensing) and also gives bad results. Thus, we use diagbox
\usepackage{diagbox}

\usepackage{xcolor}

% Code Listings
\usepackage{listings}

\definecolor{eclipseStrings}{RGB}{42,0.0,255}
\definecolor{eclipseKeywords}{RGB}{127,0,85}
\colorlet{numb}{magenta!60!black}

% JSON definition
% Source: https://tex.stackexchange.com/a/433961/9075

\lstdefinelanguage{json}{
    basicstyle=\normalfont\ttfamily,
    commentstyle=\color{eclipseStrings}, % style of comment
    stringstyle=\color{eclipseKeywords}, % style of strings
    numbers=left,
    numberstyle=\scriptsize,
    stepnumber=1,
    numbersep=8pt,
    showstringspaces=false,
    breaklines=true,
    frame=lines,
    % backgroundcolor=\color{gray}, %only if you like
    string=[s]{"}{"},
    comment=[l]{:\ "},
    morecomment=[l]{:"},
    literate=
        *{0}{{{\color{numb}0}}}{1}
         {1}{{{\color{numb}1}}}{1}
         {2}{{{\color{numb}2}}}{1}
         {3}{{{\color{numb}3}}}{1}
         {4}{{{\color{numb}4}}}{1}
         {5}{{{\color{numb}5}}}{1}
         {6}{{{\color{numb}6}}}{1}
         {7}{{{\color{numb}7}}}{1}
         {8}{{{\color{numb}8}}}{1}
         {9}{{{\color{numb}9}}}{1}
}

\lstset{
  % everything between (* *) is a latex command
  escapeinside={(*}{*)},
  %
  language=json,
  %
  showstringspaces=false,
  %
  extendedchars=true,
  %
  basicstyle=\footnotesize\ttfamily,
  %
  commentstyle=\slshape,
  %
  % default: \rmfamily
  stringstyle=\ttfamily,
  %
  breaklines=true,
  %
  breakatwhitespace=true,
  %
  % alternative: fixed
  columns=flexible,
  %
  numbers=left,
  %
  numberstyle=\tiny,
  %
  basewidth=.5em,
  %
  xleftmargin=.5cm,
  %
  % aboveskip=0mm,
  %
  % belowskip=0mm,
  %
  captionpos=b
}

% Enable Umlauts when using \lstinputputlisting.
% See https://stackoverflow.com/a/29260603/873282 für details.
% listingsutf8 did not work in June 2020.
\lstset{literate=
  {á}{{\'a}}1 {é}{{\'e}}1 {í}{{\'i}}1 {ó}{{\'o}}1 {ú}{{\'u}}1
  {Á}{{\'A}}1 {É}{{\'E}}1 {Í}{{\'I}}1 {Ó}{{\'O}}1 {Ú}{{\'U}}1
  {à}{{\`a}}1 {è}{{\`e}}1 {ì}{{\`i}}1 {ò}{{\`o}}1 {ù}{{\`u}}1
  {À}{{\`A}}1 {È}{{\'E}}1 {Ì}{{\`I}}1 {Ò}{{\`O}}1 {Ù}{{\`U}}1
  {ä}{{\"a}}1 {ë}{{\"e}}1 {ï}{{\"i}}1 {ö}{{\"o}}1 {ü}{{\"u}}1
  {Ä}{{\"A}}1 {Ë}{{\"E}}1 {Ï}{{\"I}}1 {Ö}{{\"O}}1 {Ü}{{\"U}}1
  {â}{{\^a}}1 {ê}{{\^e}}1 {î}{{\^i}}1 {ô}{{\^o}}1 {û}{{\^u}}1
  {Â}{{\^A}}1 {Ê}{{\^E}}1 {Î}{{\^I}}1 {Ô}{{\^O}}1 {Û}{{\^U}}1
  {Ã}{{\~A}}1 {ã}{{\~a}}1 {Õ}{{\~O}}1 {õ}{{\~o}}1
  {œ}{{\oe}}1 {Œ}{{\OE}}1 {æ}{{\ae}}1 {Æ}{{\AE}}1 {ß}{{\ss}}1
  {ű}{{\H{u}}}1 {Ű}{{\H{U}}}1 {ő}{{\H{o}}}1 {Ő}{{\H{O}}}1
  {ç}{{\c c}}1 {Ç}{{\c C}}1 {ø}{{\o}}1 {å}{{\r a}}1 {Å}{{\r A}}1
}

% For easy quotations: \enquote{text}
% This package is very smart when nesting is applied, otherwise textcmds (see below) provides a shorter command
\usepackage[autostyle=true]{csquotes}

% Enable using "`quote"' - see https://tex.stackexchange.com/a/150954/9075
\defineshorthand{"`}{\openautoquote}
\defineshorthand{"'}{\closeautoquote}

% Nicer tables (\toprule, \midrule, \bottomrule)
\usepackage{booktabs}

% Extended enumerate, such as \begin{compactenum}
\usepackage{paralist}

% Bibliopgraphy enhancements
%  - enable \cite[prenote][]{ref}
%  - enable \cite{ref1,ref2}
% Alternative: \usepackage{cite}, which enables \cite{ref1, ref2} only (otherwise: Error message: "White space in argument")

% Doc: http://texdoc.net/natbib
\usepackage[%
  square,        % for square brackets
  comma,         % use commas as separators
  numbers,       % for numerical citations;
  %sort           % orders multiple citations into the sequence in which they appear in the list of references;
  sort&compress % as sort but in addition multiple numerical citations
                  % are compressed if possible (as 3-6, 15);
]{natbib}

% In the bibliography, references have to be formatted as 1., 2., ... not [1], [2], ...
\renewcommand{\bibnumfmt}[1]{#1.}

% Enable hyperlinked author names in the case of \citet
% Source: https://tex.stackexchange.com/a/76075/9075
\usepackage{etoolbox}
\makeatletter
\patchcmd{\NAT@test}{\else \NAT@nm}{\else \NAT@hyper@{\NAT@nm}}{}{}
\makeatother

% Prepare more space-saving rendering of the bibliography
% Source: https://tex.stackexchange.com/a/280936/9075
\SetExpansion
[ context = sloppy,
  stretch = 30,
  shrink = 60,
  step = 5 ]
{ encoding = {OT1,T1,TS1} }
{ }

% Put figures aside a text
\usepackage[rflt]{floatflt}

% Enable nice comments
\usepackage{pdfcomment}

\newcommand{\commentontext}[2]{\colorbox{yellow!60}{#1}\pdfcomment[color={0.234 0.867 0.211},hoffset=-6pt,voffset=10pt,opacity=0.5]{#2}}
\newcommand{\commentatside}[1]{\pdfcomment[color={0.045 0.278 0.643},icon=Note]{#1}}

% Compatibality with packages todo, easy-todo, todonotes
\newcommand{\todo}[1]{\commentatside{#1}}

% Compatiblity with package fixmetodonotes
\newcommand{\TODO}[1]{\commentatside{#1}}

% Put footnotes below floats
% Source: https://tex.stackexchange.com/a/32993/9075
\usepackage{stfloats}
\fnbelowfloat

\usepackage[group-minimum-digits=4,per-mode=fraction]{siunitx}
\addto\extrasgerman{\sisetup{locale = DE}}

% Enable that parameters of \cref{}, \ref{}, \cite{}, ... are linked so that a reader can click on the number an jump to the target in the document
\usepackage{hyperref}

% Enable hyperref without colors and without bookmarks
\hypersetup{
  hidelinks,
  colorlinks=true,
  allcolors=black,
  pdfstartview=Fit,
  breaklinks=true
}

% Enable correct jumping to figures when referencing
\usepackage[all]{hypcap}

\usepackage[caption=false,font=footnotesize]{subfig}

\usepackage{mindflow}

% Extensions for references inside the document (\cref{fig:sample}, ...)
% Enable usage \cref{...} and \Cref{...} instead of \ref: Type of reference included in the link
% That means, "Figure 5" is a full link instead of just "5".
\usepackage[capitalise,nameinlink]{cleveref}

\crefname{section}{Sect.}{Sect.}
\Crefname{section}{Section}{Sections}
\crefname{listing}{List.}{List.}
\crefname{listing}{Listing}{Listings}
\Crefname{listing}{Listing}{Listings}
\crefname{lstlisting}{Listing}{Listings}
\Crefname{lstlisting}{Listing}{Listings}

\usepackage{lipsum}

% For demonstration purposes only
% These packages can be removed when all examples have been deleted
\usepackage[math]{blindtext}
\usepackage{mwe}
\usepackage[realmainfile]{currfile}
\usepackage{tcolorbox}
\tcbuselibrary{listings}

%introduce \powerset - hint by http://matheplanet.com/matheplanet/nuke/html/viewtopic.php?topic=136492&post_id=997377
\DeclareFontFamily{U}{MnSymbolC}{}
\DeclareSymbolFont{MnSyC}{U}{MnSymbolC}{m}{n}
\DeclareFontShape{U}{MnSymbolC}{m}{n}{
  <-6>    MnSymbolC5
  <6-7>   MnSymbolC6
  <7-8>   MnSymbolC7
  <8-9>   MnSymbolC8
  <9-10>  MnSymbolC9
  <10-12> MnSymbolC10
  <12->   MnSymbolC12%
}{}
\DeclareMathSymbol{\powerset}{\mathord}{MnSyC}{180}

\usepackage{xspace}
%\newcommand{\eg}{e.\,g.\xspace}
%\newcommand{\ie}{i.\,e.\xspace}
\newcommand{\eg}{e.\,g.,\ }
\newcommand{\ie}{i.\,e.,\ }

% Enable hyphenation at other places as the dash.
% Example: applicaiton\hydash specific
\makeatletter
\newcommand{\hydash}{\penalty\@M-\hskip\z@skip}
% Definition of "= taken from http://mirror.ctan.org/macros/latex/contrib/babel-contrib/german/ngermanb.dtx
\makeatother

% Add manual adapted hyphenation of English words
% See https://ctan.org/pkg/hyphenex and https://tex.stackexchange.com/a/22892/9075 for details
% Does not work on MiKTeX, therefore disabled - issue reported at https://github.com/MiKTeX/miktex-packaging/issues/271
% \input{ushyphex}

% correct bad hyphenation here
\hyphenation{op-tical net-works semi-conduc-tor}

% Add copyright
%
% This is recommended if you intend to send the version to colleagues
% See https://ctan.org/pkg/llncsconf for details
\iffalse
  % state: intended | submitted | llncs
  % you can add "crop" if the paper should be cropped to the format Springer is publishing
  \usepackage[intended]{llncsconf}

  \conference{name of the conference}

  % in case of "proceedings" (final version!)
  % example: \llncs{Anonymous et al. (eds). \emph{Proceedings of the International Conference on \LaTeX-Hacks}, LNCS~42. Some Publisher, 2016.}{0042}
  % 0042 denotes an example start page
  \llncs{book editors and title}{0042}
\fi

% Enable copy and paste of text from the PDF
% Only required for pdflatex. It "just works" in the case of lualatex.
% Alternative: cmap or mmap package
% mmap enables mathematical symbols, but does not work with the newtx font set
% See: https://tex.stackexchange.com/a/64457/9075
% Other solutions outlined at http://goemonx.blogspot.de/2012/01/pdflatex-ligaturen-und-copynpaste.html and http://tex.stackexchange.com/questions/4397/make-ligatures-in-linux-libertine-copyable-and-searchable
% Trouble shooting outlined at https://tex.stackexchange.com/a/100618/9075
%
% According to https://tex.stackexchange.com/q/451235/9075 this is the way to go
\input glyphtounicode
\pdfgentounicode=1

\begin{document}

\title{KBQA Final Report}
% If Title is too long, use \titlerunning
%\titlerunning{Short Title}

% Single insitute
\author{Firstname Lastname \and Firstname Lastname}

% If there are too many authors, use \authorrunning
%\authorrunning{First Author et al.}

\institute{Institute}

%% Multiple insitutes - ALTERNATIVE to the above
% \author{%
%     Firstname Lastname\inst{1} \and
%     Firstname Lastname\inst{2}
% }
%
%If there are too many authors, use \authorrunning
%  \authorrunning{First Author et al.}
%
%  \institute{
%      Insitute 1\\
%      \email{...}\and
%      Insitute 2\\
%      \email{...}
%}

\maketitle

% Problem or Task definition (~0.5 page)
% Technical description of the method to solve the task (~1 - 2 page)

\section{Approach Description}

% NSpM
Our basis approach is the Neural SPARQL Machine (NSpM)\cite{soru-marx-nampi2018},
which considers SPARQL query as another natural language and employ machine translation techniques. 
NSpM consists of three main components: 
Generator, Learner and Interpreter. 
Generator generates a training set
by fulling entities from knowledge graph into question and query templates with placeholders. 

Following is an example of template and generated questions and queries:

\begin{verbatim}
    Where is <A> located in?; 
                    SELECT ?x { <A> dbo:location ?x }
\end{verbatim}

\begin{verbatim}
    Where is London located in?; 
                    SELECT ?x { dbr:London dbo:location ?x }
    Where is the Colosseum located in?; 
                    SELECT ?x { :Colosseum :location ?x }
    Where is Mount Everest located in?; 
                    SELECT ?x { :Mount_Everest :location ?x }
\end{verbatim}

Learner is a deep neural network based translator, 
which translates a sequence of tokens in natural language
into a sequence which encodes a SPARQL query after training.
Interpreter uses the learned model to predict a SPARQL query for received question.

There are two drawbacks in this original NSpM.
Since training data is generated using templates, 
the number of entities are severely restricted,
which leads to a bad performance in evaluation.
Or the training set has to be very large
that makes it hard to train with. 
Besides, there is no template for QALD 8 and 9 training set,
which make comparisons difficult. 
On the other hand, 
learner consists only of a one-layer encoder and a one-layer decoder.
The performance will be greatly improved using other translation model.
To improve NSpM, we used DBpedia Spotlight and Pegasus model. 

% DBpedia spotlight
DBpedia Spotlight \cite{isem2013daiber} is a tool for automatic annotating and entity linking, 
which performs entity extraction including entity detection and name resolution. 
First, we integrated DBpedia Spotlight into the interpreter to detect entities. 
We abandoned generator and trained with templates directly. 
Then in the prediction phase, 
when interpreter receives a question, 
it detects entities in the question using DBpedia Spotlight and replaces them with placeholders.
Interpreter translates this modified question into a SPARQL query with placeholders
and restores entities again. 
Integration of DBpedia Spotlight in interpreter decreases the size of training set
and therefore the training effort is reduced. 
However, the performance is better,
since it uses the entire knowledge graph while prediction
and learner only needs to focus on learning predicates in SPARQL query. 
DBpedia Spotlight is also used for generate templates from QALD 8 and 9 training set,
finding entities in quesiton and query and replacing them with placeholders. 

% Pegasus: pre-training with extracted gap-sentences for abstractive summarization
Pegasus\cite{10.5555/3524938.3525989} is a pre-trained model with extracted gap-sentences for abstractive summarization. 
We used hugging face pegasus model, which is pre-trained on a large text corpora. 
Pegasus model enlarges our set of tokens and improved the quality of translation. 
Moreover, we pre-trained again on LC-QALD dataset and fine-tuned on QALD 8 and 9. 

% TODO: summarize our approach

% ConvS2S
We also tried Convolutional Sequence-to-Sequence model (ConvS2S) \cite{DBLP:journals/corr/GehringAGYD17} to improve the translation quality.
ConvS2S converts an input sequence to a output sequence using convolution neural networks. 
This approach did not work in the end
as we could not rewrite it in tensorflow 2. 

% Gerbil
For evaluation we used gerbil\cite{gerbil}.
Gerbil is a web-based platform for comparison of QA system. 
User can upload test set and add a QA system via URI or upload a JSON file with answers in QALD-JSON format. 
We focused on QALD 8 and 9 for evaluation. 
At the beginning, we generated a JSON file with answers and uploaded it to gerbil. 
Later once we deployed our QA system on VM, we added our system via URI every time. 

% If applicable, evaluation of the implemented tasks (~0.5 - 1 page)

\section{Evaluation}

We used gerbil for evaluation.
Gerbil can evaluate QA systems on a QALD dataset automatically,
computing micro precision, recall and F1 score, 
macro precision, recall and F1 score,
and F1 QALD score. 
Additionally, the average answering time is also calculated. 

We concentrated on QALD 8 and 9 datasets, on which we trained and evaluated. 
The questions in QALD 9 are more complicated than in QALD 8, 
e.g. multiple triples and more logic.
Also, QALD 9 dataset contains more questions.
Therefore, QALD 9 is more challenging than QALD 8 for evaluation. 

Following table shows our evaluation results from approach A during the Project Group:

\begin{tabular}{ccccccc} \hline
    \multicolumn{5}{c}{QALD-8} \\ \hline
    \textbf{Date} & \textbf{Model}  & \textbf{Precision} & \textbf{Recall} & \textbf{F1} & \textbf{F1 QALD} \\ \hline
    06.12       & NSpM          & 0         & 0         & 0         & 0         \\
    20.12       & NSpM          & 0.0244    & 0.0244    & 0.0244    & 0.0476    \\
    23.01       & NSpM\_SL       & 0.1707    & 0.1626    & 0.1602    & 0.2777    \\
    14.06       & NSpM\_PSL      & 0.2561    & 0.2683    & 0.2602    & 0.4149    \\
    27.06       & NSpM\_LCPSL    & 0.3171    & 0.3415    & \textbf{0.3252}    & \textbf{0.5025}    \\
    \hline
                & Tebaqa        & 0.4756     & 0.4878    & 0.4797   & 0.556   \\
    \hline
\end{tabular}

\begin{tabular}{ccccccc} \hline
    \multicolumn{5}{c}{QALD-9} \\ \hline
    \textbf{Date} & \textbf{Model}  & \textbf{Precision} & \textbf{Recall} & \textbf{F1} & \textbf{F1 QALD} \\ \hline
    23.01       & NSpM\_SL       & 0.1299    & 0.1344    & 0.1312    & 0.2362    \\
    14.06       & NSpM\_PSL      & 0.2479    & 0.2694	 & \textbf{0.2454}    & \textbf{0.4127}    \\
    27.06       & NSpM\_LCPSL    & 0.2283    & 0.2464    & 0.2237    & 0.3845    \\
    \hline
                & Tebaqa        & 0.2413    & 0.2452    & 0.2384    & 0.3741  \\
    \hline
\end{tabular}

At the beginning of the Project group,
in order to try out the original NSpM model and gerbil evaluation,
we trained with QALD 8 train dataset for 8 epochs. 
We also implemented a python script to simulate 
receiving natural language question, 
converting it to query,
sending request to DBpedia endpoint, 
and finally generating a dataset with answers in QALD format,
since we did not have an URI for our system until that time. 

After figuring out the functions of each component, 
we trained NSpM again for more epochs. 
This time, the evaluation result was not zero anymore, 
which proved that,
the NSpM is a fesible approach. 

By inspecting the SPARQL queries in training and test set, 
we noticed that
many entities appear in test set are not in training set. 
Therefore, we integrated DBpedia Spotlight to avoid this problem. 
With DBpedia Spotlight and more training data, 
the result had been greatly improved. 

For a better conversion from natural language question to SPARQL query, 
we used hugging face Pegasus model instead of the simple encoder and decoder neural network in the original NSpM.

At the end of the second semester, we tried pre-training on LC-QALD dataset, then fine-tuning on QALD 8 or QALD 9. 
Interestingly, the score for QALD 8 increases with pre-training, 
but for QALD 9 decreases. 
However, we did not enough time to figure out the reason behind it. 

Compare to TeBaQa \cite{DBLP:journals/corr/abs-2103-06752},
our approach overperforms TeBaQa on QALD 9 and performs close to TeBaQa on QALD 8.
% A summary of learned skills (~0.5 page)

\section{Learned Skills}

During this two semester project group,
I have learned a lot in all aspects. 

% Scrum
We used Scrum to manage our team,
I have therefore gathered experience with Scrum, 
e.g. How to be a Scrum master to organize a team, 
how to manage weekly tasks with a board, 
and what to do in meetings. 
Also, the most important thing is how to work as a team, 
since I usually worked alone previously. 

% coding
I have also improved my coding style in this project. 
Thanks to the linter, 
I wrote more readable code and never forgot to add comments to functions, 
which make it easier for others and me later to understand.

% git
I have formed good habits in git, 
committing frequently and with meaningful comments. 
I practiced git functions such as creating new branch, merging, rebasing, 
and also how to handle merge conflicts.
Additionally, I have learned how to set up CI/CD in Github from my teammates, 
which simplified our deployment.

% VM
This is my first time working with server on a virtual machine
with nginx, REST and docker. 
I think this experience can be applied on many other projects. 

% dbpeida
I got my first insight into knowledge graph in this project group
and knew how to query knowledge graph with SPARQL query. 

% NLP
The last but not least,
I learned much about natural language processing in practice, 
including encoder and decoder, tokenizer and model training. 
I had some experience in building our own dataset for a NLP model. 
Beyond neutral network, I also became acquainted with transformer,
how to fine-tune pre-trained model on downstream tasks. 
% A summary of issues (~0.5 page)

\section{Issues}

During this project group,
I have met many issues, 
but with the help of our teammates, 
the most of them were solved quickly.

% VM
At first, I did not know what to do with our virtual machine. 
I asked my teammates then figured out how to deploy our software on it. 

% gerbil
In our evaluation at beginning, 
we must create our dataset in QALD format with answer
and upload it to gerbil,
since the system was not deployed on VM at that time. 
We created a script to do it automatically. 

% dataset
Inspired by templates and generator from NSpM,
we used templates, questions and queries with placeholders, 
instead of generated questions and queries.
QALD 8 and 9 datasets does not fit our approach in training. 
Thus, we implemented a script to replace entities in QALD 8 and 9 with placeholders
and convert to the format we need. 
In order to include more predicates in our dataset, 
we also checked classes in DBpedia 
and wrote questions and queries on our own,
even though they were unused for evaluation in the end. 

% GPU cuda
While training our model on felis Server, 
I encountered some issues with incompatible cuda and driver version. 
This was fixed quickly after some research. 

% ConvS2S
At the beginning of the second semester,
we found ConvS2S model,
which is implemented in Tensorflow version 1. 
There is a big difference between Tensorflow version 1 and 2,
we have spent a lot of time to reimplement it in Tensorflow 2
and added additional features that we want.
However, after the implementation we trained ConvS2S with questions and queries,
it could not predict any query. 
We tried many ways to fix this model, 
but it did not work in the end
and we switched to Pegasus model. 

% pretrain
At the end of project group, 
we wanted to pre-train on LC-QALD and fine-tune on QALD 8 and 9,
which could not be done directly. 
After checking parameters in Pegasus model, 
we knew how to do it. 
% Short self-evaluation with a grade that you would give yourself (~0.5 page)

\section{Self-evaluation}

In this two semester I think I did a good job. 
I am glad to cooperate and help other teammates. 
If others have some problem, 
I am willing to help and find the solution together. 
I also came up with some ideas for improving our approach, 
we tried several of them, the results are good and bad.
Sometimes we went to the wrong direction, for example with NSpM, 
but in the end the result of our approach is good. 
Due to lack of practical experience with buiding neural network and training on server, 
the speed of development was slowed down. 
Therefore, I would give myself a score of 1.7. 


\end{document}
